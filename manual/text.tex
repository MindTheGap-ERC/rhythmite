\documentclass[a4paper]{article}
\usepackage[cm]{fullpage}
\usepackage{xcolor}
\usepackage{url}
\usepackage{bm}
\usepackage{listings}
\usepackage{amsmath}
\usepackage{amsfonts}

\newcommand{\nn}{\nonumber}
\newcommand{\python}{\color{darkgray} \sffamily }
\lstset{ 
  language=Python,
  backgroundcolor=\color{white},   % choose the background color; you must add \usepackage{xcolor}; should come as last argument
  basicstyle=\footnotesize,        % the size of the fonts that are used for the code
  breakatwhitespace=false,         % sets if automatic breaks should only happen at whitespace
  breaklines=true,                 % sets automatic line breaking
  captionpos=b,                    % sets the caption-position to bottom
  frame=single,                    % adds a frame around the code
  keepspaces=true,                 % keeps spaces in text, useful for keeping indentation of code 
  keywordstyle=\color{blue},       % keyword style
}


\begin{document}

%%%%%%%%%%%%%%%%%%%%%%%%%%%%%%%%%%%%%%%%%%%%%%%%%%%%%%%%%%%%%%%%%%%%%%%%%%%%%%%%%%%%%%%%%%%%%%%%%%%

\subsubsection*{Context}

In the spring of 2024 I was approached by Emilia with regards to an article she and 
her team wished to reproduce. Although recently published, and written by a renowned computational
geoscientist, this article did not include link to a code. 
The article in question is \cite{lheu18} 
Can we reproduce the results of the paper using only the equations provided in the paper?

%----------------------------
\subsubsection*{The PDEs of the physical problem}

The whole geological context is a bit lost on me so I/we approached the project with a pragmatic 
attitude: We must solve 5 coupled nonlinear advection-diffusion-reaction PDEs in a 1D domain.

The model describes the evolution of the concentrations of five variables:
\begin{itemize}
\item Solids: two minerals which dissolve and (re)precipitate from/to the same
solutes. They are expressed as proportions of the solid phase, and must
accordingly be non-negative, and their sum must be smaller or equal to one.
$C_C$ is the proportion of calcite in the solid phase, and $C_A$ is the 
proportion of aragonite in the solid phase.

\item Solutes $\hat{c}_k$, where $k \in \{ Ca,CO3\}$.
with $\hat{c}_{Ca}$: calcium concentration in pore water, and
$\hat{c}_{CO3}$: carbonate concentration in pore water.

\item Porosity $\phi$
\end{itemize}


The equations involving our five unknowns $C_A$, $C_C$, $\hat{c}_k$ and $\phi$ are as follows (in
order to highlight how nonlinear and coupled these equations are, I have colored the 
five unknowns throughout):
\begin{eqnarray}
\frac{\partial {\color{teal} C_A}}{\partial t} 
&=& -U({\color{teal}\phi}) \frac{\partial {\color{teal}C_A}}{\partial x} 
- Da[(1-{\color{teal}C_A}){\color{teal}C_A}(\Omega_{DA}-\nu_1\Omega_{PA})+
\lambda {\color{teal}C_AC_C} (\Omega_{PC}-\nu_2\Omega_{DC})]
\nn\\
\frac{\partial {\color{teal}C_C}}{\partial t} 
&=& -U({\color{teal}\phi}) \frac{\partial {\color{teal}C_C}}{\partial x}  
+ Da[(1-{\color{teal}C_C}){\color{teal}C_C}(\Omega_{PC}-\nu_2\Omega_{DC})+
\lambda {\color{teal}C_AC_C} (\Omega_{DA}-\nu_1\Omega_{PA})]
\nn\\
\frac{\partial {\color{teal}\hat{c}_k}}{\partial t} 
&=& -W({\color{teal}\phi}) \frac{\partial {\color{teal}\hat{c}_k}}{\partial x}
+\frac{1}{\color{teal}\phi} \frac{\partial}{\partial x} 
\left( {\color{teal}\phi} d_k \frac{\partial {\color{teal}\hat{c}_k}}{\partial x} \right)
+Da \frac{1-{\color{teal}\phi}}{{\color{teal}\phi}}(\delta-{\color{teal}\hat{c}_k})
[{\color{teal}C_A}(\Omega_{DA}-\nu_1\Omega_{PA})-\lambda 
{\color{teal}C_C} (\Omega_{PC}-\nu_2 \Omega_{DC})  ]
\nn\\ 
\frac{\partial {\color{teal}\phi}}{\partial t} 
&=& -\frac{\partial}{\partial x} (W {\color{teal}\phi})
+d_{\phi} \frac{\partial^2 {\color{teal}\phi}}{\partial x^2} + Da (1-{\color{teal}\phi})
[{\color{teal}C_A}(\Omega_{DA}-\nu_1 \Omega_{PA})-\lambda 
{\color{teal}C_C} (\Omega_{PC}-\nu_2 \Omega_{DC})] \nn
\end{eqnarray}
$k=1$ corresponds to $\hat{c}_{CA}$, and
and $k=2$ corresponds to $\hat{c}_{C03}$.

We assume that the coordinate $x=0$ corresponds to the top of the domain
which is of length $L$. 

The PDEs above are actually dimensionless (the primes have been dropped), with 
\[
x'=x/X^\star, \quad t'=t/T^\star \quad \hat{c}_k'=\hat{c}_k/\sqrt{K_C},
\quad U=u/S, \quad W=w/S.
\]
and with 
\[
Da=k_2 T^\star = \frac{k_2 D_{Ca}^0}{S^2},
\quad
\lambda=k_3/k_2,
\quad
\nu_1=k_1/k_2,
\quad
\nu_2=k_4/k_3,
\quad
d_k=D_k/D_{Ca}^0, 
\quad
d_\phi=D_\phi/D_{Ca}^0,
\quad
\delta=\frac{\rho_s}{\mu_A \sqrt{K_C}}.
\]
Da can be interpreted as a Damköhler number\footnote{\url{https://en.wikipedia.org/wiki/Damkohler_numbers}}. 


%......................................
\subsubsection*{Hydraulic conductivity}

\[
K(\phi)
=\beta \frac{\phi^3}{(1-\phi)^2} F(\phi)
=\beta \frac{\phi^3}{(1-\phi)^2} \left[ 1-\exp\left( -\frac{10(1-\phi)}{\phi} \right) \right] 
\]
which translated into the following code snippet:
\begin{lstlisting}
def K(self,phi):
    return self.beta*(phi**3/(1-phi)**2)*(1-np.exp(-10*(1-phi)/phi))
\end{lstlisting}

Since the porosity $\phi$ is between 0 and 1 and $K(\phi)$ contains terms which 
could become infinite within this range, we could ask ourselves:
\[
\lim_{\phi\rightarrow 0} K(\phi) = ?
\qquad
\lim_{\phi\rightarrow 1} K(\phi) = ?
\]
Let us recall the Taylor expansion of the exponential function in zero:
\[
\exp (x) \simeq 1 + x + \frac{x^2}{2} + \frac{x^3}{6} + \dots
\]
When $\phi \rightarrow 1$, then $(1-\phi)/\phi \rightarrow 0$ and then 
\[
1-\exp\left( -\frac{10(1-\phi)}{\phi} \right) \sim 1- (1 -\frac{10(1-\phi)}{\phi} ) = \frac{10(1-\phi)}{\phi}
\]
so that 
\[
\lim_{\phi\rightarrow 1} K(\phi) = \beta \frac{\phi^3}{(1-\phi)^2} \cdot \frac{10(1-\phi)}{\phi}
=10 \beta \frac{\phi^2}{1-\phi}
\]
which is Eq.~(16) of the paper.

Also, since When $\phi \rightarrow 0$, then $(1-\phi)/\phi \rightarrow +\infty$ and then
\[
1-\exp\left( -\frac{10(1-\phi)}{\phi} \right) \sim 1 - \exp -\infty = 1 
\]
and finally
\[
\lim_{\phi\rightarrow 0} K(\phi) = 0
\]

%............................
\subsubsection*{Velocities}

W is the velocity of the pore water

The velocities $U$ and $W$ are given by 
\begin{eqnarray}
U(\phi)&=& 1-\frac{K(\phi^0)}{S}(1-\phi^0)(\frac{\rho_s^0}{\rho_w}-1) 
+\frac{K(\phi)}{S}(1-\phi) (1-\phi^0)(\frac{\rho_s}{\rho_w}-1)    \\
W(\phi)&=& 1-\frac{K(\phi^0)}{S}(1-\phi^0)(\frac{\rho_s^0}{\rho_w}-1) 
-\frac{K(\phi)}{S} \frac{(1-\phi)^2}{\phi} (1-\phi^0)(\frac{\rho_s}{\rho_w}-1)   
\end{eqnarray}

These translate into
\begin{lstlisting}
def U(self, phi):
    u = 1 - ( 1 / self.sed_rate )*\
            ( self.K(self.phi_0) * ( 1 - self.phi_0 ) - self.K(phi) * ( 1 - phi ) )*\
            ( self.rho_s0 / self.rho_w - 1 )
    return u

def W(self,phi):
    w = 1-( 1 / self.sed_rate )*\
          ( self.K(self.phi_0) * ( 1 - self.phi_0 ) + self.K(phi)*(1-phi)**2/phi)*\
          ( self.rho_s0 / self.rho_w - 1 )
    return w
\end{lstlisting}

The $U$ velocity contains the term $K(\phi)(1-\phi)$
and the $W$ contains the term $K(\phi)(1-\phi)^2/\phi$.
We find
\begin{eqnarray}
\lim_{\phi\rightarrow 0} K(\phi)(1-\phi) &=& 0 \nn\\
\lim_{\phi\rightarrow 1} K(\phi)(1-\phi) &=& 10 \beta \frac{\phi^2}{1-\phi}(1-\phi) = 10 \beta\nn\\
\lim_{\phi\rightarrow 0} K(\phi)(1-\phi)^2/\phi 
&=& \beta  \frac{\phi^3}{(1-\phi)^2} (1-\phi)^2/\phi = \beta \phi^2 = 0 \nn\\
\lim_{\phi\rightarrow 1} K(\phi)(1-\phi)^2/\phi &=& 10 \beta \frac{\phi^2}{1-\phi} (1-\phi)^2/\phi = 0 \nn
\end{eqnarray}

%..............................................
\subsubsection*{Porosity diffusion coefficient}

$d_\phi$ is the dimensionless form, calculated as $d_\phi=D_\phi/D^0_{Ca}$.
In the article and in our calculations, it is set to be constant.

\begin{lstlisting}
self.d_phi = self.beta*(self.phi_init**3 / ( 1 - self.phi_init ) )*\ 
             (1 / ( self.b*self.g*self.rho_w*( self.phi_NR - self.phi_inf ) ) )*\ 
             (1 - np.exp( -10*( 1 - self.phi_init ) / self.phi_init) )*( 1 / self.D_Ca_0 )
\end{lstlisting}

%..............................................
\subsubsection*{$k$ components diffusion coefficient}

$D_k(\phi)$ the (tortuosity-corrected) diffusion
coefficient of component $k$.
In Eq.~6 of the paper we find 
\[
D_k = \frac{D_k^0}{1- \log \phi^2}
\]
where $D_k^0$ is the diffusion of component $k$ in seawater.

\begin{lstlisting}
def d_c_ca(self, phi):
        # scaled with D_Ca_0
        return 1.0/( 1 - 2*np.log(phi) )

def d_c_co(self, phi):
        # scaled with D_Ca_0
        return self.D_CO/self.D_Ca_0*(1 / ( 1 - 2*np.log(phi) ) )
\end{lstlisting}




%....................................
\subsubsection*{Saturation factors}

$\Omega_{PA}$ and $\Omega_{DA}$ are saturation factors for precipitation 
and dissolution of aragonite. 
$\Omega_{PC}$ and $\Omega_{DC}$ are saturation factors for precipitation 
and dissolution of calcite.

The under- and oversaturation factors 

\[
\Omega_{PA}=\left( \frac{\hat{c}_{CA} \hat{c}_{CO3} K_C}{K_A} -1 \right)^{m'}
\]
where $m'$ is a reaction order (different from $m$ in general).
It is understood that this reaction is only effective when the
system is oversaturated, i.e. $\hat{c}_{CA} \hat{c}_{CO3} K_C/K_A -1 >0$.

\[
\Omega_{DA}= \left(1-\frac{\hat{c}_{CA} \hat{c}_{CO3} K_C}{K_A}  \right)^m \theta(x)
\]
where $K_A$ is the solubility of aragonite and $m$ is a reaction
order. It is understood that this reaction occurs only when
the system is undersaturated,
i.e. $1-\hat{c}_{CA} \hat{c}_{CO3}/K_A >0$.
Finally,
the factor $\theta(x)$ is a characteristic function that is zero when
$x$ is outside the aragonite dissolution zone (ADZ) and one
otherwise. The ADZ is characterized by the position of its top
edge $x_d$ and its thickness $h_d$.

Note that there is a confusion about $m$ and $m'$
in eqs 27 and 45 of the paper, but in practice we set $m=m'$.

Looking at the PDEs we find that $\Omega_{DA}$ and $\Omega_{PA}$
always occur together so we define
\[
\Omega_A=\Omega_{DA}-\nu_1\Omega_{PA} 
\]
Likewise we define
\[
\Omega_C=\Omega_{PC}-\nu_2\Omega_{DC}
\]
leading to write

\begin{lstlisting}
def Omega_A(self, c_ca, c_co, x):
    sp = c_ca*c_co*self.K_C/self.K_A - 1
    Omega_PA = (max(0.0,sp))**self.m
    sa = 1 - c_ca*c_co*self.K_C/self.K_A
    Omega_DA = (max(0.0,sa))**self.m * heaviside(x,self.ADZ_bot,self.ADZ_top,self.x_scale)
    return Omega_DA - self.nu1*Omega_PA

def Omega_C(self, c_ca, c_co):
    sp = c_ca*c_co - 1
    Omega_PC = (max(0.0,sp))**self.n
    sa = 1 - c_ca*c_co
    Omega_DC = (max(0.0, sa))**self.n
    return Omega_PC - self.nu2*Omega_DC
\end{lstlisting}

%----------------------------------------------------------------------
\subsection*{Numerical methods}

we solve the coupled PDEs by the Method of Lines, i.e.
we will discretise the right hand sides of the equations by means of the Finite 
Difference Method but will keep the time derivative as they are and we will 
use an Initial Value Problem integrator, e.g. Runge-Kutta methods.

We have implemented two methods: a simple 1st order Euler method (with a 
user-chosen constant timestep $dt$), and 
the use of the {\python solve\_ivp} from {\python scipy} %(see \stone~156,157). 
The code has been designed with modularity in mind so that many functions 
have been created, each carrying out a single task.

Because sharp transitions and discontinuities are very often source of error
in numerical modelling we have replaced the Heaviside function $\theta(x)$
by a smoothed version (the value of 500 controls the smoothness):

\begin{lstlisting}
def heaviside(x,xbot,xtop,xscale):
    val=0.5*(1+np.tanh((x-xtop/xscale)*500)) *0.5*(1+np.tanh((xbot/xscale-x)*500))
    return val 
\end{lstlisting}

As the equations require many constant parameters, we created a class to contain all the functions and parameter variables.  This way the functions need only take the solution variables and the `self' keyword as arguments, rather than passing lengthy lists of parameters to each function.  Parameter variables are set in the constructor function `\_\_init\_\_(self)' then accessed within the function using `self.var\_name'.  We also applied jit (just in time) compiling to the entire class, using the Numba package \footnote{\url{https://numba.readthedocs.io/en/stable/}}.  This produces compiled binaries for python functions at the first time they are encountered in the run and then reuses these binaries for all subsequent function calls, resulting in dramatically improved performance.  For this code, speed-up varies between a factor of roughly 3-10, depending on the mode the code runs in.  The addition of jit to the class is relatively simple, the main addition is that the type of all parameter values must be explicitly specified and passed to the jit decorator as a list of 2-element tuples specifying the variable name and its type:
\begin{lstlisting}
from numba.experimental import jitclass

# for the jit-compiling, we need to specify the type of all parameters in the class
spec = [
    ('g', float64),
    ('K_C', float64),
    ('K_A', float64),
    ('ADZ_bot', float64),
    ('ADZ_top', float64),
    .
    ..
    ...
    ]
# This class contains the functions and parameters that are used to calculate
# the RHS of eqns 40-43 of L'Heureux (2018)
@jitclass(spec)
class LHeureux:
    
    # define all params as instance vars
    # this way we can easily modify them at the instance level
    # i.e. for parameter searches
    def __init__(self):
        # physical constants
        self.g = 9.81*100           # gravitational acceleration (cm/s^2)
        # model parameters
        self.K_C = 10**-6.37        # Calcite solubility (M^2)
        self.K_A = 10**-6.19        # Aragonite solubility (M^2)
        self.ADZ_top = 50           # top of the Aragonite dissolution zone (cm)
        self.ADZ_bot = 150
        .
        ..
        ...
\end{lstlisting}

\begin{eqnarray}
\frac{\partial C_A}{\partial t} &=& -U \frac{\partial C_A}{\partial x}  + {\cal R}_{\text Aragonite}
\nn\\
\frac{\partial C_C}{\partial t} &=& -U \frac{\partial C_C}{\partial x}  + {\cal R}_{\text Calcite} 
\nn\\
\frac{\partial \hat{c}_k}{\partial t} &=& -W \frac{\partial \hat{c}_k}{\partial x}
+\frac{1}{\phi} \frac{\partial}{\partial x} \left( \phi d_k \frac{\partial \hat{c}_k}{\partial x} \right)
+{\cal R}_{k}
\nn\\ 
\frac{\partial \phi}{\partial t} &=& -\frac{\partial}{\partial x} (W \phi) +{\cal R}_\phi
\end{eqnarray}

The ${\cal R}$ rhs terms are implemented via five functions:
\begin{lstlisting}
#Aragonite
def R_AR(self, AR, CA, c_ca, c_co, x):
    return - self.Da*( ( 1 - AR ) * AR * self.Omega_A(c_ca, c_co, x) +\
                        self.lamb * AR * CA * self.Omega_C(c_ca, c_co) )

#Calcite
def R_CA(self, AR, CA, c_ca, c_co, x):
    return self.Da*( self.lamb * ( 1 - CA ) * CA * self.Omega_C(c_ca, c_co) +\
                        AR * CA * self.Omega_A(c_ca, c_co, x) )

# Ca ions
def R_c_ca(self, AR, CA, c_ca, c_co, phi, x):
    return self.Da*( ( 1 - phi ) / phi ) * (self.delta - c_ca)*\
               ( AR * self.Omega_A(c_ca, c_co, x) - self.lamb * CA * self.Omega_C(c_ca, c_co) )

# CO3 ions
def R_c_co(self, AR, CA, c_ca, c_co, phi, x):
    return self.Da*( ( 1 - phi ) / phi ) * (self.delta - c_co)*\
               ( AR * self.Omega_A(c_ca, c_co, x) - self.lamb * CA * self.Omega_C(c_ca, c_co) )

# porosity
def R_phi(self, AR, CA, c_ca, c_co, phi, x):
    return self.Da*( 1 - phi )*\
               ( AR * self.Omega_A(c_ca, c_co, x) - self.lamb * CA * self.Omega_C(c_ca, c_co) )
\end{lstlisting}

The solution for all variables is stored in a single array, 5*nnx long, with the first nnx elements corresponding to the AR values, the next nnx giving CA and so on.  The full ordering of the variables in X is AR, CA, c\_ca, c\_co, phi.  This structure is designed for use with the ivp\_solve function, which expects a single array for the solution variable values.  ivp\_solve also expects a single function which calculates the RHS, so we also combine the RHS functions in a master function:
\begin{lstlisting}
def X_RHS(self, t, X, nnx, x, h):
        
dAR_dt   = self.RHS_AR(  X[0:nnx], X[nnx:2*nnx], X[2*nnx:3*nnx], X[3*nnx:4*nnx], X[4*nnx:5*nnx], x, h)
dCA_dt   = self.RHS_CA(  X[0:nnx], X[nnx:2*nnx], X[2*nnx:3*nnx], X[3*nnx:4*nnx], X[4*nnx:5*nnx], x, h)
dc_ca_dt = self.RHS_c_ca(X[0:nnx], X[nnx:2*nnx], X[2*nnx:3*nnx], X[3*nnx:4*nnx], X[4*nnx:5*nnx], x, h)
dc_co_dt = self.RHS_c_co(X[0:nnx], X[nnx:2*nnx], X[2*nnx:3*nnx], X[3*nnx:4*nnx], X[4*nnx:5*nnx], x, h)
dphi_dt  = self.RHS_phi( X[0:nnx], X[nnx:2*nnx], X[2*nnx:3*nnx], X[3*nnx:4*nnx], X[4*nnx:5*nnx], x, h)
        
return np.concatenate((dAR_dt, dCA_dt, dc_ca_dt, dc_co_dt, dphi_dt)) 
\end{lstlisting}


%----------------------------------------------------------------------
\subsection*{Initial values}

The initial conditions are chosen as spatially homogeneous constants:
\begin{eqnarray}
C_A(x,0)&=&C_{A,init} \\
C_C(x,0)&=&C_{C,init} \\
\hat{c}_k(x,0)&=&\hat{c}_{k,init} \\
\phi(x,0) &=& \phi_{init}
\end{eqnarray}
for $x\ne 0$. This is Eq.~36 in the paper.

%----------------------------------------------------------------------
\subsection*{Boundary conditions}

At $x=0$ (top of the domain):
\begin{eqnarray}
C_A(0,t)&=&C_{A}^0 \\
C_C(0,t)&=&C_{C}^0 \\
\hat{c}_k(0,t)&=&\hat{c}_{k}^0 \\
\phi(0,t) &=& \phi^0
\end{eqnarray}
At $x=L$ (bottom of the domain):
\[
\frac{\partial \hat{c}_k (x,t)}{\partial x}|_L = 0
\]
\[
\frac{\partial \phi (x,t)}{\partial x}|_L = 0
\]

we need boundary conditions there because of the second-order 
derivative in these equations (diffusion term).


%----------------------------------
\subsubsection*{1st PDE}

\[
\frac{\partial {\color{teal} C_A}}{\partial t} 
= -U({\color{teal}\phi}) \frac{\partial {\color{teal}C_A}}{\partial x} 
- Da[(1-{\color{teal}C_A}){\color{teal}C_A}(\Omega_{DA}-\nu_1\Omega_{PA})+
\lambda {\color{teal}C_AC_C} (\Omega_{PC}-\nu_2\Omega_{DC})]
\]
Turning to the Aragonite PDE, we must now produce a FD approximation of the advection term.
For all the nodes inside the domain, use use a centered approximation:
\[
-U \frac{\partial C_A}{\partial x} \rightarrow -U_i \frac{C_A|_{i+1} -C_A|_{i-1}}{2h}
\]
For the node at the top (i.e. $i=0$), we are prescribing the value of $C_A$, i.e.
the derivative is then zero so then 
\[
-U \frac{\partial C_A}{\partial x} \rightarrow 0 
\]
For the node at the bottom, no boundary condition is imposed so we use a backward
approximation:
\[
-U \frac{\partial C_A}{\partial x} \rightarrow -U_i \frac{C_A|_{i} -C_A|_{i-1}}{h}
\]
 
\begin{lstlisting}
def RHS_AR(self, AR, CA, c_ca, c_co, phi, x, h):
    dAR_dt = np.zeros(len(x))
    u = self.U(phi)
    for i in range(0,len(x)):
        # x = 0 BC, Dirichlet
        if (i==0):
            dAR_dt[i]= 0
        # x = Lx, no prescribed BC
        elif (i==len(x)-1):
            dAR_dt[i]= -u[i]*(AR[i]-AR[i-1])/h + self.R_AR(AR[i],CA[i],c_ca[i],c_co[i],x[i])
        else:
            dAR_dt[i]= -u[i]*(AR[i+1]-AR[i-1])/(2*h) + self.R_AR(AR[i],CA[i],c_ca[i],c_co[i], x[i])    
    return dAR_dt
\end{lstlisting}

%----------------------------------
\subsubsection*{2nd PDE}
\[
\frac{\partial {\color{teal}C_C}}{\partial t} 
= -U({\color{teal}\phi}) \frac{\partial {\color{teal}C_C}}{\partial x}  
+ Da[(1-{\color{teal}C_C}){\color{teal}C_C}(\Omega_{PC}-\nu_2\Omega_{DC})+
\lambda {\color{teal}C_AC_C} (\Omega_{DA}-\nu_1\Omega_{PA})]
\]

The structure of the 2nd PDE is identical to the first one so we adopt the same approch.
\begin{lstlisting}
def RHS_CA(self, AR, CA, c_ca, c_co, phi, x, h):
    dCA_dt = np.zeros(len(x))
    u = self.U(phi)
    for i in range(0,len(x)):
        # x = 0 BC, Dirichlet
        if (i==0):
            dCA_dt[i]= 0
        # x = Lx, no prescribed BC
        elif (i==len(x)-1):
            dCA_dt[i]= -u[i]*( CA[i]-CA[i-1] ) / h + self.R_CA(AR[i], CA[i], c_ca[i], c_co[i], x[i])
        else:
            dCA_dt[i]= -u[i]*( CA[i+1]-CA[i-1])/(2*h)+self.R_CA(AR[i],CA[i],c_ca[i],c_co[i],x[i])
    return dCA_dt
\end{lstlisting}

%----------------------------------
\subsubsection*{3rd PDE}

\[
\frac{\partial {\color{teal}\hat{c}_k}}{\partial t} 
= -W({\color{teal}\phi}) \frac{\partial {\color{teal}\hat{c}_k}}{\partial x}
+\frac{1}{\color{teal}\phi} \frac{\partial}{\partial x} 
\left( {\color{teal}\phi} d_k \frac{\partial {\color{teal}\hat{c}_k}}{\partial x} \right)
+Da \frac{1-{\color{teal}\phi}}{{\color{teal}\phi}}(\delta-{\color{teal}\hat{c}_k})
[{\color{teal}C_A}(\Omega_{DA}-\nu_1\Omega_{PA})-\lambda 
{\color{teal}C_C} (\Omega_{PC}-\nu_2 \Omega_{DC})  ]
\]
The advection term is treated in the same way as above, but we here have to discretise 
a diffusion term with a non constant diffusion coefficient.
The standard procedure in such cases is as follows:
\[
a
\] 



\end{document}
